\documentclass{article}
\usepackage[T1]{fontenc}
\begin{document}

\section{New Section}

\subsection{New Subsection}

Hello,World!

\section{Quick Reference}

\subsection{Intrinsic Font Styles}

\textbf{boldface}
\textit{italic} 
\textrm{roman}
\textsc{small caps}
\textsf{sans serif}
\textsl{slanted}
\texttt{typewriter}

\textbf{\textsc{bold small caps}}

\subsection{Font Sizes}
{\tiny tiny} {\scriptsize smaller still} {\footnotesize smaller} {\small small} normal {\large large} {\Large larger} {\LARGE larger still} {\huge huge}

\begin{large} This is large environment.\end{large}

\subsection{Special Format}
\underline{underline} \frame{frame} \fbox{frame with some room}

\subsection{Positioning Paragraph}
\begin{center}
The text is centered because I have entered the center environment.
Text remains centered as long as we remain in this environment.
\end{center}
\begin{flushleft}
Now we are out of the centering environment, and have begun the
flushleft environment.
\end{flushleft}
\begin{flushright}
This is another paragraph, but in the flushright environment.
You will have occasion to use all four paragraph positions.
\end{flushright}
\centerline{This line is centered.}

\subsection{Skipping}
This is a first line. \bigskip

The space you see above is a big skip. \medskip

The space you see just above is a medium skip. \smallskip

The space you now see just above is a small skip.

This is just an ordinary line space.

\subsection{Quote}
\begin{quote}
The quote environment is intended for short quotes, generally one short
paragraph (as me), or a sequence of one line quotes, separated by blank lines.
\end{quote}
\begin{quotation}
The quotation environment is used for long quotations, having more
than one paragraph (separated by blank lines).

\noindent The indentation is the same
as the quote, except the first line of each new paragraph is indented.
\end{quotation}

Another example is as followed:
\begin{quotation}
``Computers do not dream, any more than they play. We are
far from certain what dreams are good for, but we know what
they indicate: a great deal of information processing goes on far
beneath the surface of man’s purposive behavior, in ways and
for reasons that are only very indirectly reflected in his overt
activity.''

\hfill --- Alan M. Turing
\end{quotation}

\subsection{Verse}
The verse environment indents oppositely: lines after the first.
\begin{verse}
\textit{Neglect of mathematics works injury to all knowledge, since he
who is ignorant of it cannot know the other sciences or the
things of this world. And what is worse, men who are thus
ignorant are unable to perceive their own ignorance and so do
not seek a remedy.} \hfill --- Roger Bacon
\end{verse}

\subsection{List Environment}
There are three intrinsic list environments, distinguished by what appears
at the beginning of each item: number, bullet, or your description (perhaps
nothing). To illustrate, here is the use of a description list environment
to itemize steps involved in learning \LaTeX.

\subsubsection{Description List Environment}
\begin{description}
	\item [Basic Document Preparation.] Knowing how to setup ...
	\item [Making Tables.] \LaTeX~ provides a means ...
	\item [Bibliography.] Knowing how to create a bibliography ...
	\item [Mathematics.] This is the power of \LaTeX~ and one ...
	\item [Graphics.] This has progressed a great deal in the ...
	\item [Other.] There are a great many things to learn ...
\end{description}

\subsubsection{Itemize List Environment}
\begin{itemize}
	\item This is item 1 and our task has just begun. Blank lines before an item have no effect.
	\item This is item 2 and we shall limit to just this few.

A blank line within an item does create a new paragraph,
using the indentation of the itemize environment.

	\begin{itemize}
		\item A second (nested) itemized list changes the bullet and indents another level.
			\begin{itemize}
					\item A third (nested) itemized list changes the bullet and indents another level.
			\end{itemize}
	\end{itemize}
\end{itemize}

\subsubsection{Enumerate List Environment}
\begin{enumerate}
	\item This is item 1, and we are having fun.
	\item This is item 2, and it’s time to number anew.
		\begin{enumerate}
			\item Back to item 1, but we are not yet done.
			\item Two is new.
				\begin{enumerate}
					\item One again!
					\item Two (b) or knot 2b?
				\end{enumerate}
		\end{enumerate}
\end{enumerate}

\subsection{Table}
\begin{center}
\begin{tabular}{lcr}
left & center & right \\
1 & 2 & 3
\end{tabular}\medskip

Figure 1:A 2*3 Table
\end{center} \bigskip
\begin{center}
\begin{tabular}{|l|c|r|} \hline
-110 & 12000 & -130 \\ \hline
210 & -220 & 230 \\ \hline
\end{tabular} \medskip

Figure 2:A 2*3 Table with Horizontal and Vertical Lines
\end{center} \bigskip

\begin{center}
\begin{tabular}{l|cc|}
Name & Test 1 & Test 2 \\ \cline{1-1}
Bob & 67 & 72 \\
Sue & 72 & 67 \\ \cline{2-3}
\end{tabular} \medskip

Figure 3:A Table with Partially Spanning Horizontal and Vertical Lines
\end{center}

\end{document}